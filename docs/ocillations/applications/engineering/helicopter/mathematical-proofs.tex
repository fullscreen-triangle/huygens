\documentclass{article}
\usepackage{amsmath,amsfonts,amssymb,amsthm}
\usepackage{natbib}

\newtheorem{theorem}{Theorem}
\newtheorem{lemma}{Lemma}
\newtheorem{proposition}{Proposition}
\newtheorem{corollary}{Corollary}
\newtheorem{definition}{Definition}

\title{Mathematical Proofs for Oscillatory Dynamics in Multi-Stage Coaxial Rotor Systems}
\author{Anonymous}
\date{\today}

\begin{document}

\maketitle

\section{Multi-Rotor Oscillatory Coupling Analysis}

\begin{theorem}[Multi-Rotor Coupling Stability Theorem]
For a multi-stage rotor system with mechanical coupling through differential gears, the system is dynamically stable if and only if all oscillatory coupling modes satisfy the stability criterion:
\begin{equation}
\text{Re}(\lambda_i) < -\zeta_{\text{min}} \omega_i
\end{equation}
where $\lambda_i$ are the eigenvalues of the coupled system matrix and $\zeta_{\text{min}}$ is the minimum required damping ratio.
\end{theorem}

\begin{proof}
Consider the coupled multi-rotor system dynamics:
\begin{align}
I_1\ddot{\theta}_1 &= Q_{\text{engine}} - Q_1 - C_{12}(\dot{\theta}_1 - k_2\dot{\theta}_2) - C_{13}(\dot{\theta}_1 - k_3\dot{\theta}_3) \\
I_2\ddot{\theta}_2 &= C_{12}(\dot{\theta}_1 - k_2\dot{\theta}_2)/k_2 + Q_{\text{recovery}} - Q_2 \\
I_3\ddot{\theta}_3 &= C_{13}(\dot{\theta}_1 - k_3\dot{\theta}_3)/k_3 - Q_3
\end{align}

where $C_{ij}$ represents mechanical coupling stiffness and $k_i$ are gear ratios.

Linearizing around the equilibrium and introducing state variables:
$$\mathbf{x} = [\theta_1, \dot{\theta}_1, \theta_2, \dot{\theta}_2, \theta_3, \dot{\theta}_3]^T$$

The system matrix becomes:
\begin{equation}
\mathbf{A} = \begin{bmatrix}
\mathbf{0} & \mathbf{I} \\
-\mathbf{M}^{-1}\mathbf{K} & -\mathbf{M}^{-1}\mathbf{C}
\end{bmatrix}
\end{equation}

where $\mathbf{M}$, $\mathbf{K}$, and $\mathbf{C}$ are the inertia, stiffness, and damping matrices respectively.

For stability, all eigenvalues of $\mathbf{A}$ must have negative real parts. For oscillatory modes:
$$\lambda_{i,i+1} = -\zeta_i \omega_i \pm i\omega_i\sqrt{1-\zeta_i^2}$$

Stability requires $\zeta_i > 0$ for all modes, with practical stability demanding $\zeta_i > \zeta_{\text{min}}$ to ensure adequate stability margins.
\end{proof}

\begin{lemma}[Gear Ratio Coupling Effect]
For mechanically coupled rotors with gear ratios $k_i$, the effective coupling strength between modes is scaled by the gear ratio according to:
\begin{equation}
C_{\text{effective}} = C_{\text{mechanical}} \cdot \frac{1}{\sqrt{k_i k_j}}
\end{equation}
\end{lemma}

\begin{proof}
The mechanical coupling torque between rotors $i$ and $j$ is:
$$Q_{ij} = C_{ij}(\theta_i/k_i - \theta_j/k_j)$$

The effective angular displacement difference is:
$$\Delta\theta_{\text{eff}} = \theta_i/k_i - \theta_j/k_j$$

The coupling energy is:
$$E_{\text{coupling}} = \frac{1}{2}C_{ij}(\Delta\theta_{\text{eff}})^2$$

Converting to modal coordinates and normalizing by the geometric mean of the gear ratios yields the stated result.
\end{proof}

\section{Resonant Energy Transfer Analysis}

\begin{theorem}[Resonant Energy Transfer Theorem]
For a multi-stage rotor system, maximum energy transfer from the primary to secondary rotors occurs when the downwash frequency matches an integer multiple of the secondary rotor natural frequency:
\begin{equation}
\omega_{\text{downwash}} = n \cdot \omega_{\text{secondary}}, \quad n \in \mathbb{Z}^+
\end{equation}
\end{theorem}

\begin{proof}
The downwash-induced forcing on the secondary rotor can be modeled as:
\begin{equation}
F_{\text{downwash}}(t) = F_0 \sum_{n=1}^{\infty} A_n \cos(n\omega_1 t + \phi_n)
\end{equation}

where $\omega_1$ is the primary rotor frequency and $A_n$ are the harmonic amplitudes.

The secondary rotor response in frequency domain is:
\begin{equation}
H(j\omega) = \frac{1}{-\omega^2 M_2 + j\omega C_2 + K_2}
\end{equation}

The power transfer is maximum when the denominator is minimized, which occurs at resonance:
$$\omega_{\text{resonance}} = \sqrt{\frac{K_2}{M_2}} = \omega_{\text{secondary}}$$

For harmonic forcing, maximum transfer occurs when:
$$n\omega_1 = \omega_{\text{secondary}}$$

The quality factor determines the sharpness of the resonance:
\begin{equation}
Q = \frac{\omega_{\text{secondary}} M_2}{C_2}
\end{equation}

Higher Q factors provide greater energy transfer efficiency but require more precise frequency matching.
\end{proof}

\begin{corollary}[Optimal Gear Ratio for Resonant Transfer]
The optimal gear ratio for resonant energy transfer is:
\begin{equation}
k_{2,\text{opt}} = \frac{n \omega_1}{\omega_{\text{secondary,natural}}}
\end{equation}
where $n$ is the chosen harmonic number.
\end{corollary>

\section{Oscillatory Energy Cascade Analysis}

\begin{theorem}[Energy Cascade Efficiency Theorem]
For a multi-stage oscillatory energy cascade system, the total energy transfer efficiency is bounded by:
\begin{equation}
\eta_{\text{total}} \leq \prod_{i=1}^{N-1} \eta_i \cdot \left(1 - \frac{\Delta\omega_i}{\omega_i}\right)
\end{equation}
where $\eta_i$ is the efficiency of stage $i$ and $\Delta\omega_i$ represents frequency mismatch.
\end{theorem}

\begin{proof}
Consider a cascade of $N$ oscillatory energy transfer stages. The energy flow through stage $i$ is:
\begin{equation}
P_i = P_{i-1} \cdot \eta_i \cdot T_i(\omega)
\end{equation}

where $T_i(\omega)$ is the frequency-dependent transfer function:
\begin{equation}
T_i(\omega) = \frac{1}{1 + \left(\frac{\Delta\omega_i}{\omega_i}\right)^2 Q_i^2}
\end{equation}

For high-Q systems and small frequency mismatches:
$$T_i(\omega) \approx 1 - \frac{\Delta\omega_i}{\omega_i}$$

The total efficiency becomes:
\begin{equation>
\eta_{\text{total}} = \frac{P_N}{P_0} = \prod_{i=1}^{N} \eta_i T_i(\omega)
\end{equation>

Substituting the approximation and recognizing that perfect frequency matching is impossible in practice yields the stated bound.
\end{proof>

\section{Dynamic Stability Under Oscillatory Coupling}

\begin{theorem}[Coupled Rotor Stability Criterion]
A mechanically coupled multi-rotor system remains stable under oscillatory perturbations if the coupling matrix $\mathbf{C}$ satisfies:
\begin{equation}
\lambda_{\min}(\mathbf{C}) > \frac{\max_i(I_i \omega_i^2)}{2\min_j(\zeta_j)}
\end{equation}
where $\lambda_{\min}$ is the minimum eigenvalue of the coupling matrix.
\end{theorem}

\begin{proof}
The linearized equations of motion for the coupled system are:
\begin{equation}
\mathbf{M}\ddot{\mathbf{q}} + \mathbf{D}\dot{\mathbf{q}} + (\mathbf{K} + \mathbf{C})\mathbf{q} = \mathbf{0}
\end{equation}

where $\mathbf{q} = [q_1, q_2, q_3]^T$ represents generalized coordinates.

For stability, all eigenvalues of the characteristic polynomial must have negative real parts:
$$\det(s^2\mathbf{M} + s\mathbf{D} + \mathbf{K} + \mathbf{C}) = 0$$

Using the Routh-Hurwitz criterion, stability requires:
\begin{equation}
\mathbf{D} > \mathbf{0}, \quad \mathbf{K} + \mathbf{C} > \mathbf{0}
\end{equation}

The critical condition occurs when the coupling matrix eigenvalue approaches the stability boundary:
$$\lambda_{\min}(\mathbf{C}) = \max_i\left(\frac{I_i \omega_i^2}{2\zeta_i}\right)$$

This gives the stated stability criterion.
\end{proof}

\section{Autorotation Dynamics with Oscillatory Coupling}

\begin{theorem}[Coupled Autorotation Theorem]
During autorotation in a coupled multi-rotor system, the descent rate is minimized when the rotors operate in phase-locked synchronization:
\begin{equation}
\phi_2 - \phi_1 = \pm\frac{\pi}{2}, \quad \phi_3 - \phi_1 = 0
\end{equation}
where $\phi_i$ represents the phase of rotor $i$.
\end{theorem}

\begin{proof}
The total autorotative power available is:
\begin{equation}
P_{\text{auto}} = \sum_{i=1}^{3} P_i + P_{\text{coupling}}
\end{equation}

where $P_{\text{coupling}}$ represents power exchange between rotors through oscillatory coupling.

The coupling power is:
\begin{equation}
P_{\text{coupling}} = \sum_{i<j} C_{ij} \omega_i \omega_j \cos(\phi_i - \phi_j)
\end{equation}

To minimize descent rate, we maximize available power. Taking partial derivatives:
\begin{align}
\frac{\partial P_{\text{auto}}}{\partial \phi_1} &= C_{12}\omega_1\omega_2\sin(\phi_1-\phi_2) + C_{13}\omega_1\omega_3\sin(\phi_1-\phi_3) = 0 \\
\frac{\partial P_{\text{auto}}}{\partial \phi_2} &= -C_{12}\omega_1\omega_2\sin(\phi_1-\phi_2) + C_{23}\omega_2\omega_3\sin(\phi_2-\phi_3) = 0 \\
\frac{\partial P_{\text{auto}}}{\partial \phi_3} &= -C_{13}\omega_1\omega_3\sin(\phi_1-\phi_3) - C_{23}\omega_2\omega_3\sin(\phi_2-\phi_3) = 0
\end{align}

Solving this system with the constraint that $C_{13} >> C_{12}, C_{23}$ (strong coupling between primary and pusher propellers) yields the stated phase relationships.
\end{proof}

\section{Vibration Isolation and Energy Recovery}

\begin{theorem}[Optimal Vibration Isolation Theorem]
For a multi-rotor system with vibration isolation, the optimal isolation frequency is:
\begin{equation}
\omega_{\text{isolation}} = \sqrt{\frac{2\omega_{\text{rotor}}^2}{1 + \sqrt{1 + 4\eta_{\text{recovery}}^2}}}
\end{equation}
where $\eta_{\text{recovery}}$ is the energy recovery efficiency target.
\end{theorem}

\begin{proof}
The vibration isolation system can be modeled as a two-degree-of-freedom system:
\begin{align}
m_1\ddot{x}_1 + c(\dot{x}_1 - \dot{x}_2) + k(x_1 - x_2) &= F_{\text{rotor}} \\
m_2\ddot{x}_2 - c(\dot{x}_1 - \dot{x}_2) - k(x_1 - x_2) + k_2 x_2 &= 0
\end{align}

The transmissibility is:
\begin{equation}
T(\omega) = \left|\frac{k_2 X_2}{F_{\text{rotor}}}\right|
\end{equation}

The energy recovered is proportional to the relative motion:
\begin{equation}
P_{\text{recovered}} \propto |X_1 - X_2|^2 \omega^3
\end{equation>

Optimizing the trade-off between isolation (minimizing $T(\omega)$) and energy recovery (maximizing relative motion) yields the stated optimal frequency.
\end{proof}

\section{Structural Resonance Avoidance}

\begin{theorem}[Resonance Avoidance Theorem]
For a multi-rotor system to avoid structural resonance, the rotor frequencies must satisfy:
\begin{equation}
\left|\frac{n_i \omega_i - \omega_{\text{struct},j}}{|\omega_{\text{struct},j}|}\right| > \frac{2}{\sqrt{Q_j}}, \quad \forall i,j,n_i
\end{equation}
where $\omega_{\text{struct},j}$ are structural natural frequencies, $n_i$ are harmonic numbers, and $Q_j$ are structural mode quality factors.
\end{theorem}

\begin{proof}
The structural response to harmonic forcing at frequency $n_i\omega_i$ is:
\begin{equation}
H_j(n_i\omega_i) = \frac{1}{1 - \left(\frac{n_i\omega_i}{\omega_{\text{struct},j}}\right)^2 + j\frac{n_i\omega_i}{\omega_{\text{struct},j} Q_j}}
\end{equation}

The response amplitude is:
\begin{equation}
|H_j| = \frac{1}{\sqrt{\left[1 - \left(\frac{n_i\omega_i}{\omega_{\text{struct},j}}\right)^2\right]^2 + \left(\frac{n_i\omega_i}{\omega_{\text{struct},j} Q_j}\right)^2}}
\end{equation}

For the response to remain bounded (avoiding resonance), we require:
$$|H_j| < H_{\text{max}}$$

For high-Q systems, the critical condition occurs near resonance where:
$$1 - \left(\frac{n_i\omega_i}{\omega_{\text{struct},j}}\right)^2 \approx 0$$

The minimum safe frequency separation is determined by setting the denominator greater than a threshold value, yielding the stated criterion.
\end{proof>

\bibliographystyle{unsrt}
\bibliography{references}

\end{document}
