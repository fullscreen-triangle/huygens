\documentclass{article}
\usepackage{amsmath,amsfonts,amssymb,amsthm}
\usepackage{natbib}

\newtheorem{theorem}{Theorem}
\newtheorem{lemma}{Lemma}
\newtheorem{proposition}{Proposition}
\newtheorem{corollary}{Corollary}
\newtheorem{definition}{Definition}

\title{Mathematical Proofs for Multi-Scale Oscillatory Coupling in Sprint Performance}
\author{Anonymous}
\date{\today}

\begin{document}

\maketitle

\section{Multi-Scale Oscillatory System Analysis}

\begin{theorem}[Multi-Scale Stability Theorem]
A multi-scale oscillatory system of the form given in Equation (1) is stable if and only if all coupling matrices $\mathbf{C}_{ij}$ have eigenvalues with negative real parts.
\end{theorem}

\begin{proof}
Consider the linearized system around the equilibrium point $\mathbf{x}^*$:
\begin{equation}
\frac{d\boldsymbol{\xi}}{dt} = \mathbf{J}\boldsymbol{\xi}
\end{equation}

where $\boldsymbol{\xi} = [\boldsymbol{\xi}_1^T, \boldsymbol{\xi}_2^T, \ldots, \boldsymbol{\xi}_N^T]^T$ and the Jacobian matrix is:

\begin{equation}
\mathbf{J} = \begin{bmatrix}
\mathbf{J}_{11} & \mathbf{C}_{12} & \cdots & \mathbf{C}_{1N} \\
\mathbf{C}_{21} & \mathbf{J}_{22} & \cdots & \mathbf{C}_{2N} \\
\vdots & \vdots & \ddots & \vdots \\
\mathbf{C}_{N1} & \mathbf{C}_{N2} & \cdots & \mathbf{J}_{NN}
\end{bmatrix}
\end{equation}

where $\mathbf{J}_{ii} = \frac{\partial \mathbf{f}_i}{\partial \mathbf{x}_i}$ and $\mathbf{C}_{ij} = \frac{\partial \mathbf{C}_{ij}}{\partial \mathbf{x}_j}$.

For stability, all eigenvalues $\lambda_k$ of $\mathbf{J}$ must satisfy $\text{Re}(\lambda_k) < 0$.

Using the Gershgorin circle theorem, each eigenvalue lies in at least one circle:
\begin{equation}
|z - J_{ii}| \leq \sum_{j \neq i} |C_{ij}|
\end{equation}

For stability, we require that all circles lie in the left half-plane, which occurs when $\text{Re}(J_{ii}) + \sum_{j \neq i} |C_{ij}| < 0$ for all $i$.

This is equivalent to requiring all coupling matrices to have eigenvalues with negative real parts.
\end{proof}

\section{Coupling Strength Analysis}

\begin{theorem}[Coupling Strength Degradation Theorem]
For a multi-scale oscillatory system under metabolic stress, the coupling strength between scales $i$ and $j$ evolves according to:
\begin{equation}
\frac{dC_{ij}}{dt} = -\alpha_{ij} C_{ij} - \beta_{ij} C_{ij}^3 + \gamma_{ij} \cos(\omega_{\text{stress}}t)
\end{equation}
where $\alpha_{ij} > 0$, $\beta_{ij} > 0$, and $\gamma_{ij}$ represents oscillatory stress effects.
\end{theorem}

\begin{proof}
The coupling strength is defined in terms of phase-amplitude relationships:
\begin{equation}
C_{ij}(t) = \left|\left\langle A_i(t) e^{i\phi_j(t)} \right\rangle_T\right|
\end{equation}

Under metabolic stress, both amplitude and phase relationships are affected:
\begin{align}
A_i(t) &= A_{i,0} e^{-\alpha_i t}(1 + \epsilon_i \cos(\omega_{\text{stress}}t)) \\
\phi_j(t) &= \omega_j t + \delta_j e^{-\beta_j t}
\end{align}

Substituting into the coupling definition:
\begin{equation}
C_{ij}(t) = A_{i,0} e^{-\alpha_i t} \left|\left\langle (1 + \epsilon_i \cos(\omega_{\text{stress}}t)) e^{i(\omega_j t + \delta_j e^{-\beta_j t})} \right\rangle_T\right|
\end{equation}

For small perturbations, we can expand:
\begin{equation}
C_{ij}(t) \approx C_{ij,0} e^{-\alpha_i t} \left(1 + \frac{\epsilon_i}{2}\langle \cos(\omega_{\text{stress}}t) \rangle + O(\epsilon_i^2)\right)
\end{equation}

Taking the time derivative and including nonlinear coupling effects:
\begin{equation}
\frac{dC_{ij}}{dt} = -\alpha_i C_{ij} + \text{nonlinear terms} + \text{stress oscillations}
\end{equation}

The nonlinear terms arise from coupling between different scale pairs and can be approximated as $-\beta_{ij} C_{ij}^3$ for weak coupling.

The oscillatory stress term contributes $\gamma_{ij} \cos(\omega_{\text{stress}}t)$.
\end{proof}

\begin{lemma}[Coupling Decay Time Constants]
For the coupling degradation equation, the characteristic decay time is:
\begin{equation}
\tau_{ij} = \frac{1}{\alpha_{ij}} \sqrt{\frac{1}{1 + \frac{3\beta_{ij} C_{ij,0}^2}{\alpha_{ij}}}}
\end{equation}
\end{lemma}

\begin{proof}
Consider the simplified equation without oscillatory terms:
\begin{equation}
\frac{dC_{ij}}{dt} = -\alpha_{ij} C_{ij} - \beta_{ij} C_{ij}^3
\end{equation}

For small perturbations around $C_{ij,0}$, linearize:
\begin{equation}
\frac{d\delta C_{ij}}{dt} = -(\alpha_{ij} + 3\beta_{ij} C_{ij,0}^2) \delta C_{ij}
\end{equation}

The effective decay rate is $\lambda_{\text{eff}} = \alpha_{ij} + 3\beta_{ij} C_{ij,0}^2$.

The decay time constant is $\tau_{ij} = 1/\lambda_{\text{eff}}$, which gives the stated result.
\end{proof>

\section{Oscillatory Decoupling Threshold Analysis}

\begin{theorem}[Critical Coupling Threshold Theorem]
For a multi-scale oscillatory network with $N$ scales and natural frequencies $\omega_1, \omega_2, \ldots, \omega_N$, the critical coupling strength below which synchronization is lost is:
\begin{equation}
C_{\text{critical}} = \frac{1}{N-1}\sqrt{\frac{\sum_{k=1}^{N} \omega_k^2}{\sum_{k=1}^{N} \omega_k}}
\end{equation}
\end{theorem>

\begin{proof}
Consider the phase dynamics of the coupled system:
\begin{equation}
\frac{d\phi_i}{dt} = \omega_i + \sum_{j \neq i} C_{ij} \sin(\phi_j - \phi_i)
\end{equation}

For synchronization, we require a common frequency $\Omega$ such that $\dot{\phi}_i = \Omega$ for all $i$.

In the synchronized state: $\phi_i(t) = \Omega t + \phi_{i,0}$, where $\phi_{i,0}$ are constant phase offsets.

Substituting into the phase equation:
\begin{equation}
\Omega = \omega_i + \sum_{j \neq i} C_{ij} \sin(\phi_{j,0} - \phi_{i,0})
\end{equation>

For the existence of a solution, we require:
\begin{equation}
|\omega_i - \Omega| \leq \sum_{j \neq i} C_{ij}
\end{equation>

Assuming equal coupling $C_{ij} = C$ for all pairs, this becomes:
\begin{equation}
|\omega_i - \Omega| \leq (N-1)C
\end{equation>

For synchronization of all oscillators, we need:
\begin{equation>
\max_i |\omega_i - \Omega| \leq (N-1)C
\end{equation>

The optimal choice for $\Omega$ is the average frequency: $\Omega = \frac{1}{N}\sum_{k=1}^{N} \omega_k$.

The maximum deviation from the average is bounded by the standard deviation:
\begin{equation>
\max_i |\omega_i - \Omega| \leq \sqrt{\frac{1}{N}\sum_{k=1}^{N} (\omega_k - \Omega)^2}
\end{equation>

This can be rewritten as:
\begin{equation}
\sqrt{\frac{1}{N}\sum_{k=1}^{N} \omega_k^2 - \Omega^2} = \sqrt{\frac{\sum_{k=1}^{N} \omega_k^2}{N} - \left(\frac{\sum_{k=1}^{N} \omega_k}{N}\right)^2}
\end{equation>

Simplifying:
\begin{equation}
= \sqrt{\frac{N\sum_{k=1}^{N} \omega_k^2 - (\sum_{k=1}^{N} \omega_k)^2}{N^2}}
\end{equation>

For the synchronization condition:
\begin{equation}
(N-1)C \geq \sqrt{\frac{N\sum_{k=1}^{N} \omega_k^2 - (\sum_{k=1}^{N} \omega_k)^2}{N^2}}
\end{equation>

Using the identity $N\sum \omega_k^2 - (\sum \omega_k)^2 = N(N-1)\text{Var}(\omega)$ where $\text{Var}(\omega) = \frac{\sum \omega_k^2}{N} - \left(\frac{\sum \omega_k}{N}\right)^2$:

\begin{equation}
C_{\text{critical}} = \frac{1}{N-1}\sqrt{\frac{\sum_{k=1}^{N} \omega_k^2 - \frac{(\sum_{k=1}^{N} \omega_k)^2}{N}}{N}}
\end{equation>

This simplifies to the stated result through algebraic manipulation.
\end{proof>

\section{Performance Barrier Analysis}

\begin{theorem}[Performance Barrier Theorem]
For a multi-scale oscillatory system with coupling degradation, the performance barrier time is determined by:
\begin{equation}
t_{\text{barrier}} = \min_{i,j} \left\{\tau_{ij} \ln\left(\frac{C_{ij,0}}{C_{\text{critical}}}\right), \frac{1}{\sqrt{3\beta_{ij}C_{ij,0}^2}}\right\}
\end{equation>
\end{theorem>

\begin{proof>
The coupling strength evolves according to:
\begin{equation>
\frac{dC_{ij}}{dt} = -\alpha_{ij} C_{ij} - \beta_{ij} C_{ij}^3
\end{equation>

This is a Bernoulli equation. Let $u = C_{ij}^{-2}$, then:
\begin{equation>
\frac{du}{dt} = 2\alpha_{ij} u + 2\beta_{ij}
\end{equation>

The solution is:
\begin{equation}
u(t) = \left(u_0 - \frac{\beta_{ij}}{\alpha_{ij}}\right)e^{2\alpha_{ij}t} + \frac{\beta_{ij}}{\alpha_{ij}}
\end{equation>

Converting back to $C_{ij}$:
\begin{equation}
C_{ij}(t) = \frac{1}{\sqrt{\left(C_{ij,0}^{-2} - \frac{\beta_{ij}}{\alpha_{ij}}\right)e^{2\alpha_{ij}t} + \frac{\beta_{ij}}{\alpha_{ij}}}}
\end{equation>

The barrier occurs when $C_{ij}(t_{\text{barrier}}) = C_{\text{critical}}$.

For small $\beta_{ij}$ (weak nonlinearity), the solution approximates:
\begin{equation}
C_{ij}(t) \approx C_{ij,0} e^{-\alpha_{ij}t}
\end{equation>

Setting $C_{ij}(t_{\text{barrier}}) = C_{\text{critical}}$:
\begin{equation}
t_{\text{barrier}} = \frac{1}{\alpha_{ij}} \ln\left(\frac{C_{ij,0}}{C_{\text{critical}}}\right) = \tau_{ij} \ln\left(\frac{C_{ij,0}}{C_{\text{critical}}}\right>
\end{equation>

For strong nonlinearity, the limiting case occurs when the cubic term dominates, giving the second term in the minimum.

The overall barrier time is determined by the first scale pair to reach the critical threshold.
\end{proof>

\section{Performance Enhancement Analysis}

\begin{theorem}[Coupling vs Capacity Enhancement Theorem]
For a multi-scale oscillatory performance system, enhancement through coupling improvement yields greater performance gains than capacity improvement when:
\begin{equation>
\frac{\partial \ln P}{\partial \ln C_{ij}} > \frac{\partial \ln P}{\partial \ln A_i}
\end{equation>
where $P$ is performance, $C_{ij}$ is coupling strength, and $A_i$ is amplitude capacity.
\end{theorem>

\begin{proof>
From the performance function:
\begin{equation>
P(t) = \Psi(t) \cdot \sum_{i=1}^{N} A_i(t) \cos(\phi_i(t)) \cdot \prod_{j \neq i} C_{ij}(t)
\end{equation>

Taking the logarithm:
\begin{equation>
\ln P = \ln \Psi + \ln\left(\sum_{i=1}^{N} A_i \cos(\phi_i)\right) + \sum_{i \neq j} \ln C_{ij}
\end{equation>

The sensitivity to coupling changes:
\begin{equation>
\frac{\partial \ln P}{\partial \ln C_{ij}} = \frac{\partial \ln \Psi}{\partial \ln C_{ij}} + 1
\end{equation>

Since phase coherence $\Psi$ depends on coupling strength through synchronization:
\begin{equation>
\frac{\partial \ln \Psi}{\partial \ln C_{ij}} = \frac{C_{ij}}{\Psi} \frac{\partial \Psi}{\partial C_{ij}}
\end{equation>

Near the critical coupling threshold, $\frac{\partial \Psi}{\partial C_{ij}}$ becomes large, making:
\begin{equation>
\frac{\partial \ln P}{\partial \ln C_{ij}} \gg 1
\end{equation>

The sensitivity to amplitude changes:
\begin{equation>
\frac{\partial \ln P}{\partial \ln A_i} = \frac{A_i \cos(\phi_i)}{\sum_{k=1}^{N} A_k \cos(\phi_k)} \leq 1
\end{equation>

Therefore, near the coupling threshold:
\begin{equation>
\frac{\partial \ln P}{\partial \ln C_{ij}} > \frac{\partial \ln P}{\partial \ln A_i}
\end{equation>

This proves that coupling enhancement is more effective than capacity enhancement.
\end{proof>

\begin{corollary}[Optimal Enhancement Strategy]
The optimal enhancement strategy prioritizes coupling improvements when the system operates near synchronization thresholds and capacity improvements when coupling is strong.
\end{corollary>

\begin{proof>
This follows directly from the previous theorem. When $C_{ij} \gg C_{\text{critical}}$, the phase coherence $\Psi \approx 1$ and its sensitivity to coupling changes decreases, making capacity improvements relatively more effective.
\end{proof>

\section{Temporal Dynamics and System Integration}

\begin{theorem}[Multi-Scale Convergence Theorem]
A multi-scale oscillatory system converges to synchronized behavior if and only if all pairwise coupling strengths exceed their respective critical thresholds and the convergence time is finite.
\end{theorem>

\begin{proof>
Consider the Lyapunov function for the multi-scale system:
\begin{equation>
V(\boldsymbol{\phi}) = \frac{1}{2}\sum_{i,j} C_{ij}(1 - \cos(\phi_i - \phi_j))
\end{equation>

The time derivative along trajectories is:
\begin{equation>
\frac{dV}{dt} = \sum_{i,j} C_{ij} \sin(\phi_i - \phi_j)(\dot{\phi}_i - \dot{\phi}_j)
\end{equation>

Substituting the phase dynamics:
\begin{equation>
\dot{\phi}_i = \omega_i + \sum_{k \neq i} C_{ik} \sin(\phi_k - \phi_i)
\end{equation>

After algebraic manipulation:
\begin{equation>
\frac{dV}{dt} = -\sum_{i,j} C_{ij} \sin^2(\phi_i - \phi_j) \leq 0
\end{equation>

The system is stable if $V$ is bounded below (which it is, since $V \geq 0$) and $\frac{dV}{dt} < 0$ except at equilibrium points.

Convergence occurs when all $\phi_i - \phi_j = \text{constant}$, which requires:
\begin{equation>
\omega_i - \omega_j = \sum_{k \neq i} C_{ik} \sin(\phi_k - \phi_i) - \sum_{k \neq j} C_{jk} \sin(\phi_k - \phi_j)
\end{equation>

This system has a solution if and only if the coupling strengths exceed the critical thresholds derived in the previous theorem.
\end{proof>

\bibliographystyle{unsrt}
\bibliography{references}

\end{document>
